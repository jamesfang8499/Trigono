\documentclass[b5paper, openany]{ctexbook}
\usepackage{zref-abspage}
\usepackage[margin=2.5cm]{geometry}


\usepackage{pifont}
\usepackage[perpage,symbol*]{footmisc}
\DefineFNsymbols{circled}{{\ding{192}}{\ding{193}}{\ding{194}}
{\ding{195}}{\ding{196}}{\ding{197}}{\ding{198}}{\ding{199}}{\ding{200}}{\ding{201}}}
\setfnsymbol{circled}



\usepackage{amsmath,amsfonts,mathrsfs,amssymb}
\usepackage{graphicx}

\usepackage[font=bf,labelfont=bf,labelsep=quad]{caption}

\usepackage{tikz}


\usepackage{ntheorem}
\theoremseparator{\;}



\usepackage{blkarray}
\usepackage{bm}
\usepackage[colorlinks=true, linkcolor=black]{hyperref}



\theoremstyle{plain}
\theoremheaderfont{\normalfont\bfseries} 
\theorembodyfont{\normalfont}


\usepackage[framemethod=tikz]{mdframed}


\newtheorem{example}{\bf 例}[chapter]
\newenvironment{solution}{\noindent {\bf 解:}}{}
\newenvironment{analyze}{\noindent {\bf 分析:}}{}
\newenvironment{rmk}{ {\bf 注意:}}{}
\newenvironment{note}{ {\bf 说明:}}{}
\newenvironment{remark}{ {\bf 评述:}}{}

\renewcommand{\emptyset}{\varnothing}

\renewcommand{\proofname}{\bf 证明:}
\newenvironment{proof}{{\noindent \bf 证明:}}{}%{\hfill $\square$\par}

\newcommand{\E}{\mathbb{E}}
\renewcommand{\Pr}{\mathbb{P}}

\newcommand{\dif}{\,{\rm d}}
\newcommand{\Var}{{\rm Var}}
\newcommand{\Cov}{{\rm Cov}}
\newcommand{\x}{\times}
\renewcommand{\Longrightarrow}{\;\Rightarrow\;}
\newcommand{\map}[3]{#1:\; #2\mapsto #3}
\newcommand{\arccot}{{\rm arccot\,}}


 \usepackage{tcolorbox}
 \tcbuselibrary{breakable}
 \tcbuselibrary{most}



\newtcolorbox{ex}[1][]
  {colback = white, colframe = cyan!75!black, fonttitle = \bfseries,
    colbacktitle = cyan!85!black, enhanced,
    attach boxed title to top center={yshift=-2mm},breakable, 
    title=练习, #1}

\newtcolorbox{blk}[1][]
  {colback = white, colframe = magenta!75!black, fonttitle = \bfseries,
    colbacktitle = magenta!85!black, enhanced,
    attach boxed title to top left={xshift=5mm, yshift=-2mm},breakable, 
    title=思考题, #1}

\newtcolorbox{thm}[2][]
  {colback = white, colframe = magenta!75!black, fonttitle = \bfseries,
    colbacktitle = magenta!85!black, enhanced,
    attach boxed title to top left={xshift=5mm, yshift=-2mm},breakable, 
    title=#2, #1}

% \newtcolorbox{note}[1][]
%   {colback = white, colframe = blue!75!black, fonttitle = \bfseries,
%     colbacktitle = blue!85!black, enhanced,
%     attach boxed title to top left={xshift=5mm, yshift=-2mm},breakable, 
%     title=说明, #1}



\setcounter{tocdepth}{1}

\setcounter{secnumdepth}{2}



% \ctexset {
% section = {
% 	name = {第,节},
%  	number = \chinese{section}},
% subsection = {
% 	name = {,、\hspace{-1em}},
% 	number = \chinese{subsection}
% },
% subsubsection = {
% 	name = {(,)\hspace{-1em}},
% 	number = \chinese{subsubsection},
% }
% }



\renewcommand{\contentsname}{目~~录}

\newcommand{\poly}{\polynomial[reciprocal]}
\newcommand{\Q}{\mathbb{Q}}
\newcommand{\R}{\mathbb{R}}
\newcommand{\N}{\mathbb{N}}
\newcommand{\Z}{\mathbb{Z}}


\usepackage{mathtools}

\setlength{\abovecaptionskip}{0.cm}
\setlength{\belowcaptionskip}{-0.cm}

\usetikzlibrary{decorations.pathmorphing, patterns}
\usetikzlibrary{calc, patterns, decorations.markings}
\usetikzlibrary{positioning, snakes, hobby}

\usepackage{lscape}

\usepackage{yhmath}
\usepackage{longdivision}
\usepackage{polynom}
\usepackage{polynomial}
\usepackage{cancel}

\renewcommand{\frac}{\dfrac}
\newcommand{\oc}{$^{\circ}{\rm C}$}
\newcommand{\blank}{\underline{\qquad}}

\usepackage{multicol}
\usepackage{cases}
% \usepackage{enumitem}
\usepackage{ulem}
\usepackage{enumerate}

\DeclareSymbolFont{ugmL}{OMX}{mdugm}{m}{n}
\DeclareMathAccent{\widearc}{\mathord}{ugmL}{"F3}

\usepackage{tkz-euclide}
\newcommand{\VEC}{\overrightarrow}
\usetikzlibrary{decorations.pathmorphing}


\begin{document}

\begin{tikzpicture}[>=stealth]
    \draw[->](-1.5,0)--(1.5,0)node[below]{$x$};
    \draw[->](0,-2.25)--(0,2.5)node[left]{$y$};
\draw[domain=-pi/2:pi/2, smooth, very thick]plot({sin(\x r)},\x);
\draw[dashed](-1,0)node[above]{$-1$}--(-1,-pi/2)--(0,-pi/2)node[right]{$-\frac{\pi}{2}$};
\draw[dashed](1,0)node[below]{$1$}--(1,pi/2)--(0,pi/2)node[left]{$-\frac{\pi}{2}$};
\node [below right]{$O$};
\end{tikzpicture}

\begin{tikzpicture}[>=stealth]
  \draw[->](-1.5,0)--(1.5,0)node[below]{$x$};
  \draw[->](0,-.5)--(0,4)node[left]{$y$};
\draw[domain=0:pi, smooth, very thick]plot({cos(\x r)},\x);
\draw[dashed](-1,0)node[below]{$-1$}--(-1,pi)--(0,pi)node[right]{$\pi$};
\node at (1,0)[below]{$1$};
\node [below left]{$O$};
\node at (0,pi/2)[above right]{$\frac{\pi}{2}$};
\end{tikzpicture}

\begin{tikzpicture}[>=stealth]
  \draw[->](-3,0)--(3,0)node[below]{$x$};
  \draw[->](0,-2)--(0,2)node[left]{$y$};
\draw[domain=-1.25:1.25, smooth, very thick]plot({tan(\x r)},\x);
\draw[dashed](-3,pi/2)--(3,pi/2);
\draw[dashed](-3,-pi/2)--(3,-pi/2);

\node [below right]{$O$};
\node at (0,-pi/2)[above right]{$-\frac{\pi}{2}$};
\node at (0,pi/2)[below left]{$\frac{\pi}{2}$};
\end{tikzpicture}

\begin{tikzpicture}[>=stealth]
  \draw[->](-3,0)--(3,0)node[below]{$x$};
  \draw[->](0,-.5)--(0,4)node[left]{$y$};
\draw[domain=.35:2.8, smooth, very thick]plot({1/tan(\x r)},\x);
\draw[dashed](-3,pi/2)--(3,pi/2);
\draw[dashed](-3,pi)--(3,pi);

\node [below left]{$O$};
\node at (0,pi/2)[below left]{$\frac{\pi}{2}$};
\end{tikzpicture}





\end{document}







\noindent
\begin{minipage}{.45\textwidth}
  \centering
\begin{tikzpicture}[>=stealth]
\draw[->](-2.5,0)--(2.5,0)node[below]{$x$};
\draw[->](0,-2.5)--(0,2.5)node[left]{$y$};
\draw(0,0)node [below left]{$O$}circle(1.5);



\end{tikzpicture}
\captionof{figure}{}
\end{minipage}
\hfill
\begin{minipage}{.45\textwidth}
  \centering
\begin{tikzpicture}[>=stealth]
\draw[->](-2.5,0)--(2.5,0)node[below]{$x$};
\draw[->](0,-2.5)--(0,2.5)node[left]{$y$};
\draw(0,0)node [below left]{$O$}circle(1.5);



\end{tikzpicture}
\captionof{figure}{}
\end{minipage}
